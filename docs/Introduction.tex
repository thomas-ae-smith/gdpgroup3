\section{Introduction}
High-speed internet has given us the capability to stream television via our computers and mobile devices, giving huge possibilities into how programmes and adverts may be presented to people. Historically, TV has divided into channels, with fixed schedules and the choice of an increasing number of channels. Channels eventually became regional, allowing some targeting of programmes (e.g., regional news) and adverts. Faster internet brought the possibility of catch-up TV, allowing people to watch programmes from the past any time they wanted.

In this report, we discuss the problems with viewer engagement that these forms of live media present, specifically how lack of relevance to the viewer can reduce engagement. This particularly affects adverts, where many different adverts shown in a short space of time leads to an increased likeliness of material of little or no interest being shown to the viewer. We study how improving relevance through granular targeting and providing feedback through interactivity can improve this engagement.

We first conducted a survey of 68 individuals to establish the former claim: how do viewers consume live media and how does adverts affect the engagement in viewers? This study demonstrated that almost 50\% of individuals claimed to pay little or no attention to adverts presented on live television. 75\% of those individuals considered relevance of adverts a contributing factor to this, with the ability to interact, skip or the volume of the advert being other factors. Furthermore, during advert breaks, viewers would be more likely than any other time to loose TV as a primary focus and perform another action, such as leaving the room.

From these, we considered two questions. First, would improving the relevance to adverts through granular targeting increase the number of individuals who pay attention to adverts? Could the addition of interactivity to adverts keep TV as the viewer's primary focus and prevent viewers from being distracted? We developed a platform that enables adverts to be granularly targeted towards individuals with an interactive overlay that may be developed by the advert publisher. 

We demonstrate this platform in practice through our product, your4.tv, a TV streaming service in which programmes and adverts are targeted towards viewers specifically using age, gender, occupation, location and their viewing habits. This is a web application which runs on tablet computers, making use of the rich forms of interactive that the large touch screen display offers.

%Current Internet TV streaming services - such as BBC iPlayer\footnote{\url{http://www.bbc.co.uk/iplayer/}} and 4oD\footnote{\url{http://www.channel4.com/programmes/4od}} - allow users to watch either Live TV or programmes that have been shown previously. This can be done via a web-interface on a computer. iPlayer and 4oD both have apps available for other platforms, such as the Sony PS3 and Android smart phones\footnote{\url{http://apps.channel4.com/view-all-apps/}, \url{https://play.google.com/store/apps/details?id=bbc.iplayer.android}}.

%The customer, inqb8r, runs a service called Project4, which similar to above, is a TV streaming service which also offers previously shown programmes. The service is aimed at students (specifically designed to be used on JA.NET) and shows Channel 4, four of its sister channels - E4, More4, 4Music and Film4 - and its own channel, studentTV.

%Additionally, adverts presented on 4od are often unrelated to the programme[cite] and can be repeated often in the same break. There is no evidence to suggest that these adverts are targeted towards users based on their viewing behaviour[check], suggesting that adverts may not be complimenting the interests of the viewer.

This project has been the result of a project proposal (Appendix~\ref{sec:appendix_proposal}) by inqb8r for a TV services targeted at tablet computers that would make use of complimentary services, such as interactive overlays and supplementary statistics. inqb8r are responsible for the development of Project4, a TV streaming service targeted towards students on JA.NET.

\subsection{The Team}

The team consists of five people, all completing Part 4 of Computer Science MEng~/~Software Engineering MEng. Since the team had previously worked together on a scripting project, we already knew each others key strengths and weaknesses, allowing us to allocate tasks effectively from the very start of the project. The team members are:
\begin{enumerate}
\item \textit{Peter West} (team leader): knowledgeable in user-interface design; strong experience with JavaScript and PHP.
\item \textit{Dexter Lowe}: worked with enterprise systems; strong experience with Java; very familiar with Eclipse.
\item \textit{Jim Skinner}: knowledgeable in machine-learning; strong experience with Python.
\item \textit{Tom Smith}: experienced with data visualisation; strong experience with JavaScript.
\item \textit{Adam Thomas}: experienced web developer; strong experience with JavaScript and PHP.
\end{enumerate}


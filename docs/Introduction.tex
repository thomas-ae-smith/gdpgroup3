\section{Introduction}
High-speed internet has given us the capability to stream television via our computers and mobile devises, giving huge possibilities into how programmes and adverts may be presented to people.

Historically, TV has been viewed on channels, with fixed schedules and the choice of an increasing number of channels. Channels could eventually be regional, allowing some targeting of programmes (e.g., regional news) and adverts. Faster internet brought the possibility of catch-up TV, allowing people to watch programmes from the past any time they wanted.

Current Internet TV streaming services - such as BBC iPlayer\footnote{\url{http://www.bbc.co.uk/iplayer/}} and 4oD\footnote{\url{http://www.channel4.com/programmes/4od}} - allow users to watch either Live TV or programmes that have been shown previously. This can be done via a web-interface on a computer. iPlayer and 4oD both have apps available for other platforms, such as the Sony PS3 and Android smartphones[check].

The customer, inqb8r, runs a service called Project4, which similar to above, is a TV streaming service which also offers previously shown programmes. The service is aimed at students (specifically designed to be used on JA.NET) and shows Channel 4, four of its sister channels - E4, More4, 4Music and Film4 - and its own channel, studentTV.

However, the burden is on the user to choose a programme [cite] in the

Additionally, adverts presented on 4od are often unrelated to the programme[cite] and can be repeated often in the same break. There is no evidence to suggest that these adverts are targeted towards users based on their viewing behaviour[check], suggesting that adverts may not be complimenting the interests of the viewer.

In this report, we study how our product, your4.tv, allows adverts to be targeted towards users, and in doing so, use the time more effectively to show adverts that users may be interested in. Users are given a TV channel which is tailored to them - they do not need to choose a programme, it is targeted towards them

We first perform a survey, in which we asked over 60 participants how they consume adverts in different forms of media.

\subsection{The Team}

The team consists of five people, all completing Part 4 of Computer Science MEng~/~Software Engineering MEng. Since the team had previously worked together on a scripting project, we knew each others key strengths and weaknesses, allowing us to allocate tasks effectively from the start of the project. The team members are:
\begin{enumerate}
\item \textit{Peter West} (team leader): knowledgeable in user-interface design; strong experience with JavaScript and PHP.
\item \textit{Dexter Lowe}: worked with enterprise systems; strong experience with Java; very familiar with Eclipse.
\item \textit{Jim Skinner}: knowledgeable in machine-learning; strong experience with Python.
\item \textit{Tom Smith}: experienced with data visualisation; strong experience with JavaScript.
\item \textit{Adam Thomas}: experienced web developer; strong experience with JavaScript and PHP.
\end{enumerate}


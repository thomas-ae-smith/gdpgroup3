\section{Introduction}
High-speed internet has given us the capability to stream television via our computers and mobile devices, giving huge possibilities into how programmes and adverts may be presented to people. It has enabled a shift from the traditional format of channels, which are composed of fixed schedules, to catch-up TV, which gives users a broad choice of shows to watch when they choose. 

Furthermore, allowing users to choose the programmes they watch brings the opportunity for broadcasters to gather information about individuals, which may be used to target individuals with programmes and advertisements. For example, YouTube\footnote{YouTube -- \footurl{http://www.youtube.com}} provides both advertisements and recommendations based on a user's activity, where 60\% of video click are as a result of recommendations \citep{davidson2012}. Despite these advances, \citet{three-screen} demonstrates that live TV remains the most popular method of video consumption.

In this report, we discuss the problems with viewer engagement that these forms of live media -- live TV in particular -- present. We specifically study how lack of relevance to the viewer can reduce engagement, as described by \citet{plummer2006measures}. This particularly affects adverts, where many different adverts shown in a short space of time leads to an increased likeliness viewer losing interest. Informed by studies of interactivity on websites described in \citet{Teo2003281}, we study how improving relevance through granular targeting and providing feedback through interactivity can improve this engagement.

We first conducted a survey of 67 individuals to establish the former claim: how do viewers consume live media and how does adverts affect the engagement in viewers? This study demonstrated that almost 50\% of individuals claimed to pay little or no attention to adverts presented on live television. 75\% of those individuals considered relevance of adverts a contributing factor to this, with the ability to interact, skip or the volume of the advert being other contributing factors. Furthermore, during advert breaks, 87\% of participants claimed they would be more likely than any other time to loose TV as a primary focus. These participants claimed they would use this as an opportunity to perform another task, such as leaving the room to make a drink.

From these results, we considered two questions. First, would improving the relevance of adverts to the viewer through granular targeting increase the number of individuals who pay attention to adverts? Secondly, could the addition of interactivity to adverts keep TV as the viewer's primary focus and prevent viewers from being distracted? We developed a platform that enables adverts to be granularly targeted towards individuals with an interactive overlay that may be developed by the advert publisher. 

We demonstrate this platform in practice through our product, \textit{Your4.tv}, a TV streaming service in which programmes and adverts are targeted towards viewers specifically based on age, gender, occupation, location and their viewing habits. This is a web application which runs on tablet computers, making use of the rich forms of interactive that the large touch screen display offers.

%Current Internet TV streaming services - such as BBC iPlayer\footnote{\url{http://www.bbc.co.uk/iplayer/}} and 4oD\footnote{\url{http://www.channel4.com/programmes/4od}} - allow users to watch either Live TV or programmes that have been shown previously. This can be done via a web-interface on a computer. iPlayer and 4oD both have apps available for other platforms, such as the Sony PS3 and Android smart phones\footnote{\url{http://apps.channel4.com/view-all-apps/}, \url{https://play.google.com/store/apps/details?id=bbc.iplayer.android}}.

%The customer, inqb8r, runs a service called Project4, which similar to above, is a TV streaming service which also offers previously shown programmes. The service is aimed at students (specifically designed to be used on JA.NET) and shows Channel 4, four of its sister channels - E4, More4, 4Music and Film4 - and its own channel, studentTV.

%Additionally, adverts presented on 4od are often unrelated to the programme[cite] and can be repeated often in the same break. There is no evidence to suggest that these adverts are targeted towards users based on their viewing behaviour[check], suggesting that adverts may not be complimenting the interests of the viewer.

\subsection{The team}

The team consists of five people, all completing Part 4 of Computer Science MEng~/~Software Engineering MEng. Since the team had previously worked together on a scripting project, we already knew each others key strengths and weaknesses, allowing us to allocate tasks effectively from the very start of the project:

\begin{description}

\item[Peter West] (team leader) is knowledgeable in user-interface design and has strong experience with JavaScript and PHP. Peter has been involved in significant web development projects, and during this project would lead the project and be a lead programmer and advisor for systems using web architecture.

\item[Dexter Lowe] has had significant experience in commercial environments and is an experienced Java developer. During this project, Dexter would use his familiarity with Eclipse IDE, to configure the streaming software, Wowza, in order to deliver video using industry-standard techniques.

\item[Jim Skinner] is knowledgeable in AI and machine-learning with strong experience with Python. During this project, Jim would be in charge of designing and implementing the recommender engines and data structures for both TV shows and adverts, which would allow relevant video to be targeted to viewers.

\item[Tom Smith] has experience with data storage and visualisation and a strong experience with JavaScript. During this project, Tom would be in charge of designing and executing the user studies and visualising data that is captured from users of the system.

\item[Adam Thomas] is an experienced web developer who has been involved in large web-development projects. With a strong experience with JavaScript and PHP, during this project Adam would be a lead programmer in front-end design and development. 

\end{description}
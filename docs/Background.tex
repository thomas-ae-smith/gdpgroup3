\section{Background}

	Mobile devices and high-speed internet has dramatically changed the way the people can view video media. In this section, we'll first discuss the popularity of different methods of watching video media and how modern methods have affected how people view video advertisements. We'll then discuss how streaming video over internet opens up opportunities for granular targeting of video media, including advertisements and TV shows. Following on from this, methods of facilitating programme recommendation will be discussed. Finally, current research in interactive video media will be reviewed, specifically in the context of improving viewer engagement.

	\subsection{History \& Recent Trends of Video Advertising}

	Typically, live television channels and streaming services display programme content with video advertisements apportioned amongst this content at both regular intervals, and between distinguishable items of content. Advertisements typically vary in subject and style in a number of ways in order to increase the efficiency of the advert by maximising the size of the captured audience who fit the target audience for the given advert.

	Television broadcasts typically have very low granularity - as they are, by their nature, viewed simultaneously by everyone who is watching a particular channel. This leaves a limited amount of factors for advertisers to maximise their efficiency. These factors include the time of day; the adjacently showing content; the typical target audience of the channel; and location at a very broad scope (regions, e.g. East Midlands).

	% Advertisers like targetting
	With television advertising costs increasing and an ever emerging online advertising market, broadcasters are noting dropping TV advert effectiveness levels. In 2012, 53\% of advertisers cited targeting as the most valuable aspect of online video. Online video typically boasts significantly more factors to target upon, including browsing history (behavioural data), user demographics and interests amongst others. On the contrary, in 2012 advertisers expressed a 41\% drop compared to the previous year in the important of reach (size of audience). This can be attributed to an increase comfort that online video has a large enough audience, which in turn emphasises the importance on targeting specific audiences. This is emphasised by 35\% of advertisers agreeing that demographics data is the most valuable form of targeting -- the largest targeting type in comparison to others such as contextual and geographic. \citep{brightroll-report}

	% Highly useful google research areas: http://www.google.co.uk/adwords/watchthisspace/benchmarks-and-insights/
	% They like good pricing models

	Google, an online advertising leader, has reinforced this shift in ideology towards targeted video by supporting a ``Cost Per View'' (CPV) pricing strategy. Cost per view differs from cost per impression (CPI) in that pricing is based upon users who choose to watch a video instead of video that is simply passively playing. This strategy can also incorporate advert skipping. YouTube, a Google product, offers ``TrueView'' which only charges an advertiser if the viewer reaches the end of the advert without using a skip button \citep{trueview}. This is an attractive option for advertisers, as TV advertising does not return any metrics to identify who out of the audience reach is engaged by the advertisement.

	% Advertisers need people to be engaged in their adverts
	Television advertisers are further concerned by the apparent loss of engagement in this form of passive advertising. \citet{three-screen} reports a growing trend in which viewers choose the best medium available (television, online and mobile) available to them to consume content according to the viewers perception of the quality of the content and its availability. 31\% of internet activity occurs when consumers are also watching television in domestic environments. A growing number of households are adopting Digital Video Recorder (DVR) technology, with 29\% of US homes able to time shift television which allows viewing only content the viewer is interested in, including the skipping of adverts. However, live TV viewing still accounts for the majority of video consumption, but is growing at a slower rate than the richer online platforms. \citep{three-screen}

	% Here comes rich video!
	The introduction of online video and touch mobile devices has introduced opportunities for novel interactivity which can not exist on TV. Online advertisements have long used animated images and flash technology to introduce this level of interactivity to the user. Recent developments in web standards such as HTML5 have given rise to rich video. Rich video allows the inclusion of out of band metadata (information separate to the video itself) in videos and the opportunity for viewers to manipulate aspects of the video advert. For example, YouTube\footnote{\footurl{http://www.youtube.com}} allows content owners to annotate their videos with extra clickable information at specific points in time. A study by the advertising network DoubleClick highlighted how this form of user engagement can increase advert efficiency, revealing ``Rich Media with Video'' to be the best choice for brand awareness, brand favour and purchase intent \citep{rich-video}. This rich interaction between the user and the device allows for advertisers to design fun, memorable and informative interactions into their adverts, and collect consequential user information through the audience to advertiser information flow.

	% Stats
	Traditional TV viewing statistics are not immediate or accurate due to the lack of feedback directly from the viewing device. This information is usually obtained by surveying small samples of viewers using the ``Nielsen Rating'' system. This system is critically flawed in that the sample consists only of viewers who have chosen to accept the offer to take part, which is not a truly random factor. This sample is also extremely small at approximately 37,000 homes in the USA which accounts for less than 0.5\% of the population \citep{nielsen-sample}.

	In contrast, online platforms allow immediate return of accurate statistics such as click through rates and viewership demographics. Advertisers consider better metrics to be the most important factor on encouraging higher digital video advertising spend.

	\subsubsection{Mobile Video}

	Smartphone usage in the UK has risen year on year. As of 2012, 51\% of the UK population owns a smartphone device. As mobile advertising is a strong growth industry, it is of interest to content owners to expose their products and services on this platform. By the pervasive nature of mobile devices, more users are relying on their smartphone outside of the home in order to research and purchase products; watch video; communicate and to stay informed. 66\% of smartphone users watch video on their device and 84\% of users notice mobile ads while using their phone. \citep{mobile-planet} These two factors together present an opportunity for content owners to engage users with their services by employing the use of rich video with advertising. With over 31\% of advertisers considering mobile video as the area in which advertising spend will increase most \citep{brightroll-report}, TV broadcasters need to shift towards a more engaging platform.


	\subsection{Targeting}


	\subsection{Programme Recommendation}

	% What are they? Are they used?
	Recommendation systems---systems which recommend unseen items to users based off relevant data---are a mature area of research, with a huge number of industrial implementations existing to back up a large theoretical base. Recommendation systems are widely used in internet streaming media services, including YouTube\footnote{\footurl{www.youtube.com}}, 4od\footnote{\footurl{www.channel4.com/programmes/4od}} and iPlayer\footnote{\footurl{www.bbc.co.uk/iplayer}}, and can be of enormous commercial value; evidence of this is given by the US\$1,000,000 2009 Netflix prize awarded to Pragmatic Chaos for improving upon Netflix's movie recommendation algorithm Cinematch, in return for granting Netflix a non-exclusive licence \citep{pragmatic_chaos}.

	% Collaborative/content-based filtering
	Two distinct approaches of recommendation system exist; collaborative-filtering and content-based filtering. Collaborative-filtering approaches attempt to build a profile of user behaviours, and recommend items which users with similar profiles have positively rated. This approach requires no data on the actual items being recommended, but requires collection of user preference data. As an example, a collaborative-filtering based programme recommender may recommend programmes to a user which similar users (users who have similar rating profiles to the user being recommended to) have rated highly, but the user has not yet seen. Content-based filtering requires information on the items, where items are recommended to a user which are similar to those which the user has positively rated in the past. The quality of recommendations weighs heavily on the quality of the item information available. An example content-based filtering programme recommender may give each programme a set of genres, and recommend programmes to a user which have similar genre sets as those they have rated highly.

	% Implicit data collection methods.
	To recommend programmes to a user, user preference data is used to construct a user preference model, from which new programmes are compared against. This preference data is often obtained through explicitly asking the user for programme ratings; while this is sufficient information to learn a users preferences, the user is required to halt their workflow to provide a rating \citep{implicit_indicators}, which may be seen as an additional chore with no immediate apparent reward. This lack of perceived benefit can lead to users being unwilling to provide explicit programme ratings \citep{8_challenges}, meaning a recommender system has no information from which to improve. In addition, explicit ratings are prone to biases from user subjectivity, item popularity and rating habits \citep[p.~304]{recommender-systems-handbook}. Collecting ratings implicitly combats this, which may be done for any detectable user interaction, sequence of interactions, or lack thereof \citep{implicit_indicators}, assuming that it correlates with user preferences.

	% Cold-start problem.
	Certain recommendation system implementations are faced with the cold start problem, which is the problem of the recommender being unable to give high-quality recommendations on items for which sufficient information has not yet been gathered. In a live TV recommendation system, at the point when a TV show is recommended to users it has not yet been aired and therefore cannot use collaborative data to recommend the programme. After airing, the show may pick up many viewer ratings, but in a system that deals only with live TV, these ratings are now useless as the programme will not be shown and hence not have the chance to be recommended again. New users also present a cold-start problem, as they start with no rating data from which to base recommendations. Designing recommender systems to deal with the cold start problem is an active area of research; \citep{cold-start-problem} have shown that that certain machine-learning methods are more able to learn from sparse data than others, despite having lower prediction accuracy with dense data, and hence switching technique depending on the density of data available may outperform a single algorithm . Approaches which combine both user rating information (collaborative data) and item information (content data) have been developed \citep{generative_models}, which can use content data as an initial source of information from which early recommendations may be based, and improve recommendations once collaborative data becomes available.

	Research is being performed into the extension of programme recommenders into advertisement recommenders \citep{contextual_advertising}. While similar algorithms are used, advert recommenders have a different set of considerations in providing optimal adverts: who is watching, what is being watched, programme popularity and business rules (e.g., the amount paid for the advertisement) \citep{contextual_advertising}.

	\subsection{Interactive Television}

	% use of interactivity in tv and advertising
	% See An Integrated Approach to Interactive and Personalized TV Advertising

	% use of social media in tv
	% See TriggerTV: Exploiting Social User Journeys within an Interactive TV System

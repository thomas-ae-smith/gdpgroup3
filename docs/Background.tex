\section{Background}

	\subsection{Television Broadcasting \& Online Video}

	Typically, live television channels and streaming services display programme content with video advertisements apportioned amongst this content at both regular intervals, and between distinguishable items of content. Advertisements typically vary in subject and style in a number of ways in order to increase the efficiency of the advert by maximising the captured audience who fit the target audience for the given advert.

	Television broadcasts typically have very low granularity - as they are, by their nature, viewed simulataneously by everyone who is watching a particular channel. This leaves a limited amount of factors for advertisers to maximise their efficiency. These factors include the time of day; the adjancently showing content; the typical target audience of the channel; and location at a very broad scope (regions, e.g. East Midlands).

	With telvision advertising costs increasing and an ever emerging online advertising market, broadcasters are noting dropping TV ad effectiveness levels. In 2012, 53\% of advertisers cited targetting as the most valuable aspect of online video. Online video typically boasts significantly more factors to target upon, including browsing history (behavioural data), user demographics and interests amongst others. On the contrary, in 2012 advertisers expressed a 41\% drop compared to the previous year in the important of reach (size of audience) which is attributed to an increase comfort that online video has a large enough audience, which in turn emphasises the importance on targetting specific audiences. This is emphasised by 35\% of advertisers agreeing that demopgrahics data is the most valuable form of targetting - the largest targetting type in comparison to others such as contextual and geographic. \citep{brightroll-report}

	% Highly useful google research areas: http://www.google.co.uk/adwords/watchthisspace/benchmarks-and-insights/

	Google, an online advertising leader, has reinforced this shift in ideology towards targetted video by supporting a ``Cost Per View'' (CPV) pricing strategy. Cost per view differs from cost per impression (CPI) in that pricing is based upon users who choose to watch a video instead of video that is simply passively playing. This strategy can also incorporate advert skipping. YouTube, a Google product, offers ``TrueView'' which only charges an advertiser if the viewer reaches the end of the advert wuthout using a skip button \citep{trueview}. This is an attractive option for advertisers, as TV advertising does not return any metrics to identify who out of the audience reach is engaged by the advertisement.

	Television advertisiers are further concerned by the apparant loss of egagement in this form of passive advertising. Nielson reports a growing trend in which viewers choose the best medium available (television, online and mobile) available to them to consume content according to the viewers perception of the quality of the content and its availability. 31\% of internet activity occurs when consumers are also watching television. A growing number of households are adopting Digital Video Recorder (DVR) technology, with 29\% of US homes able to timeshift television which allows viewing only content the viewer is interested, including the skipping of adverts. However, live TV viewing still accounts for the majority of video consumption, but is growing at a slower rate than the richer online platforms. \citep{three-screen}


	\subsection{Inqb8r}

	Inqb8r\footnote{\footurl{http://inqb8r.tv}} is a UK based business which provides a platform for content owners and advertisers alike to expose their own products and services to a university student focused audience. Inqb8r also provides content creation services to their customers in order to aid them in maximising their contact with students.

	\subsubsection{Project4}

	Project4 is an addressable advertising service

	Younger viewers in the student age range (18-24) also appear to watch online video at a higher rate than others...


	\subsection{Interactive Advertising}

	Advertising has traditionally been static imagery or prerecorded video with, particularly in the case of TV advertising, a broad target audience, coarse targeting and minimal feedback, with all information flowing from the advertiser to the audience. The rise of internet media streaming services has suddenly added two new possibilities: the possibility of an information flow from the audience back to the advertiser, and the possibility of a user on a device such as a laptop to easily interact with adverts. This plane of interactivity allows for advertisers to design fun, memorable and informative interactions into their adverts, and collect user information through the audience$\rightarrow$ advertiser information flow.

	stuff on better campaign tools/metrics from brightrooll



	\subsection{Programme Recommendation}

	Recommendation systems are a mature area of research, with a huge variety of implementations existing to back up a large theoretical base. Blah blah...

	% Implicit data collection methods.

	% Cold-start problem.
	Collaborative-filtering based recommendation systems are faced with the cold start problem, which is the problem of the recommender being unable to give high-quality recommendations on items for which sufficient information has not yet been gathered. In a live TV recommendation system, at the point when a TV show is recommended to users, it has not yet been aired and therefore cannot use collaborative data to recommend the programme. After airing, the show may pick up many viewer ratings, but in a system that deals only with live TV, these ratings are now useless as the programme will not be shown and hence not have the chance to be recommended again. Designing recommender systems to deal with the cold start problem is an active area of research, and studies have shown that item-based algorithms outperform SVD-based algorithms in early stages of recommendation\citep{cold-start-problem}. In addition, collaborative approaches have been combined with content-based to provide an initial source of information\citep{generative_models}.

	\subsection{Mobile devices}

	brightroll report has some evry inetresting stats on this


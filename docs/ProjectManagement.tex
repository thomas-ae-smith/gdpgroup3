\section{Methodology}

The project would be completed over a 10 week period, so it was essential that the project be started promptly. During the first week on the project the team met with the customer (represented by Peter Wood and Nick Culley) and supervisor, Les Carr, the specification that the customer had provided was discussed and the primary tasks of the project identified. This was key to agreeing on a project that would be completed successfully given the time and resource constraints.

Having established a project, the team compiled a brief (see Appendix~\ref{sec:appendix_brief}) that described the deliverable product and research focus. We also drew up a time plan in the form of a Gantt Chart (see Appendix~\ref{sec:appendix_gantt}), which we could refer to throughout the project to ensure we were on track.

\subsection{Task Allocation}
It was clear that the project could be divided into separate tasks:
\begin{enumerate}
\item Video streaming - requires understanding of streaming and use of Java\\
		Dexter
\item Programme and advert recommendation - machine learning, use of datasets\\
		Jim
\item iPad client - HTML5 video, JavaScript, CSS\\
		Peter and Adam
\item Statistics visualisation - JavaScript\\
		Tom
\end{enumerate}

\subsection{Project Management}

\begin{enumerate}
\item Leader - ensures that the team work well together by allocating tasks effectively, scheduling meetings and milestones and ensuring that resources are available where they are needed. Ensures the project remains within the agreed constraints of scope and time limit and is responsible for keeping the supervisors and customers informed of progress.
\item Architect - understands and oversees how the system integrates together and ensures it is well documented. If a team member has any doubt in the architecture, it is up to the architect to deliver an answer or brief the team if any changes are required. Must oversee areas where subsystems interact. This primarily involved ensuring that internal APIs are agreed, well designed and thoroughly documented.
\item Lead tester - responsible for ensuring that all parts of the system are tested adequately by use of software testing technique and that all testing has been documented. While they may not necessarily perform all testing themselves, they should have confidence that all parts of the system are well tested in order to avoid failure of the system as a whole.
\item Analyst - keeps the product on track to the specification agreed with the customer and supervisors. Performs an ongoing risk assessment and informs the architect's designs. Should ensure that all parts of the system are documented with regard to the customer's specifications - i.e., how certain parts cover points in the specification. Must ensure that project remains within scope and does not exceed time and resource constraints.
\item Documenter - responsible for ensuring that documentation is kept throughout the project. They take minutes in meetings and from other communications and ensure that all documentation is well organised in preparation for final write-up at the end of the project. They should ensure that documentation complements the expected standard of delivery (i.e. the mark scheme).
\end{enumerate}

\subsection{Meetings}

The team met several times a week. First in the form of a progress meeting, which would on a occasion also be attended by the supervisors and customer. Frequent progress meetings ensures that the team is on-track, by allowing each member to explain what they've accomplished and how they've documented it ensuring that the project is on schedule.

\subsection{Risk Analysis}

Resources unavailable

Network outage

Injury

Person unavailable -> group programming (more that one person knows each part of the system)

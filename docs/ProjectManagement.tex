\section{Methodology}

The project would be completed over a 10 week period, so it was essential that the project be started promptly. During the first week on the project the team met with the customer (represented by Peter Wood and Nick Culley) and supervisor, Les Carr, the specification that the customer had provided was discussed and the primary tasks of the project identified. This was key to agreeing on a project that would be completed successfully given the time and resource constraints.

Having established a project, the team compiled a brief (see Appendix~\ref{sec:appendix_brief}) that described the deliverable product and research focus. We also drew up a time plan in the form of a Gantt Chart (see Appendix~\ref{sec:appendix_gantt}), which we could refer to throughout the project to ensure we were on track.

\subsection{Project management}

For this project, it was agreed that it an agile methodology would be appropriate, where the primary tasks are agreed but remained flexible, allowing requirements to change. This decision was further motivated by the lack of experience with Wowza, on which the project's success would depend, and the unknown availability of iPads, which would be required for testing. Several studies needed to be scheduled to elicit information -- an agile approach allowed us to adjust our goals based on the outcome of these studies. A more rigid approach could have led to tasks blocking project progression and a less constructive use of the resources available to us.

A source control system was used to allow the team to work on a large codebase. The decision was made to used Mercurial, as it allows the creation of branches in which individuals can work on separate components of the system. By using BitBucket\footnote{Bitbucket: Source Control Provider -- \footurl{https://bitbucket.org/}} as a central repository, we mitigate the likeliness of deletion of work (they have reliable backing up\footnote{Bitbucket Reliability -- \footurl{https://confluence.atlassian.com/pages/viewpage.action?pageId=288658413}}) and allow work to be restored if something goes wrong. Bitbucket repositories include issue tracking such that the group could keep track of bugs and enhancements.

\subsubsection{Team Roles}

In order to ensure that the tasks involved in the project were kept on track, each group member was designated a role with particular responsibilities:

\begin{description}
\item[Leader (Peter)] ensures that the team work well together by allocating tasks effectively, scheduling meetings and milestones and ensuring that resources are available where they are needed. Ensures the project remains within the agreed constraints of scope and time limit and is responsible for keeping the supervisors and customers informed of progress.
\item[Architect (Dexter)] understands and oversees how the system integrates together and ensures it is well documented. If a team member has any doubt in the architecture, it is up to the architect to deliver an answer or brief the team if any changes are required. Must oversee areas where subsystems interact. This primarily involves ensuring that internal APIs are agreed, well designed and thoroughly documented.
\item[Lead tester (Jim)] responsible for ensuring that all parts of the system are tested adequately by use of software testing technique and that all testing has been documented. While they may not necessarily perform all testing themselves, they should have confidence that all parts of the system are well tested in order to avoid failure of the system as a whole.
\item[Analyst (Adam)] keeps the product on track to the specification agreed with the customer and supervisors. Performs an ongoing risk assessment and informs the architect's designs. Should ensure that all parts of the system are documented with regard to the customer's specifications - i.e., how certain parts cover points in the specification. Must ensure that project remains within scope and does not exceed time and resource constraints.
\item[Documenter (Tom)] responsible for ensuring that documentation is kept throughout the project. They take minutes in meetings and from other communications and ensure that all documentation is well organised in preparation for final write-up at the end of the project. They should ensure that documentation complements the expected standard of delivery.
\end{description}

The team met several times a week. First in the form of a progress meeting, which would on a occasion also be attended by the supervisors and customer. Frequent progress meetings ensures that the team is on-track, by allowing each member to explain what they've accomplished and how they've documented it ensuring that the project is on schedule.
 
The team would also meet for a weekly developing session, where the team would split into two/three groups to develop parts of the programme. Working in groups ensures that architecture and code can be understood by more than one person, reducing the concentration of essential knowledge, and mitigating the problems caused by any particular team member being out of action / uncontactable for whatever reason.

%\newpage
\begin{center}
\begin{landscape}
 \begin{longtable}{>{\raggedright}p{3cm} >{\raggedright}p{2cm} >{\raggedright}p{7cm} p{11cm}}
 \toprule
 \textbf{Hazard} &
 \textbf{Likliness} &
 \textbf{Severity} &
 \textbf{Mitigation}\\
     \toprule
        Team member illness or incapacitation. &
        Moderate &
		If a team member falls ill then our production capability will be reduced. If they are in charge of a blocking issue then all progress may be impeded. &
		All tasks performed by a team member will be overseen by another team member such that if a member is incapacitated, work may continue in that area. \\ 
     \midrule
        Team conflict &
        Moderate &
		If the team is not harmonious the friction could reduce performance and result in a lack of progression.  &
		Peter West was unanimously elected leader and chairman of the team. The team will be encouraged to bring any disputes up at the weekly meeting where all members may hear about issues. Each member will submit an anonymous vote to the chairman who will inform the group of the result. If the member still feels unhappy they may ask the group supervisor to offer an opinion which may result in a re-vote. In case of equal votes, the chairman will have a deciding vote. \\ 
     \midrule
        Project direction disagreement (Within the team) &
        High &
		If two members disagree on an aspect of the project neither may wish to continue until the issue has been resolved which may impede progression.  &
		There will be at least one meeting of the entire group per week which the supervisors and clients will be invited to attend. In this meeting the team will discuss the options and decide on a direction. In the case on an unresolvable disagreement the chairman has the deciding vote. \\ 
     \midrule
        Project direction disagreement (With supervisor) &
        Moderate &
		If the group fails to perform as the supervisors expect this may result in a poor mark. &
		The project direction is determined by the group, the client and the supervisor. If there is a disagreement with the supervisor then an emergency meeting will be called to resolve the issue as soon as possible, reducing the time spent on unprofitable activities. If the group feels that the supervisor is being unreasonable they may approach in turn their other supervisor or the module leader to resolve the issue.  \\ 
     \midrule
        Project direction disagreement (With customer) &
        Moderate &
		The group design project is supposed to be based on an external client's requirements therefore if there is a disagreement with the client this may impact the usefulness of the project to the client and hence the mark for the project due to poor communication.  &
		The team will apprise the representative of the client, Peter Wood of any changes in direction to assure that the client remains satisfied with the project and its progress. In addition the client will be invited to all meetings and presentations. Finally, the group has obtained contact information for Peter Wood and if necessary may contact him to arrange a meeting to discuss direction shift or progress. \\ 
     \midrule
        Learning curve too steep  &
        Moderate &
		If the group decides to use a technology they are not familiar with there is the possibility that it will take longer to become sufficiently knowledgeable than is allowable. &
		Where possible the team will use technologies the members are familiar with rather than using new ones. In the case where this is not possible a knowledgeable mentor will be sought. Group coding will also be encouraged to concentrate group expertise in the same room. \\ 
     \midrule
        Learning curve too steep (Wowza) &
        High &
		This project requires the use of the Java based Wowza Media Server software. None of the team have experience with this technology and little experience of streaming. As the project is heavily based on streaming it is vital that the team have access to a working Wowza server otherwise this will heavily restrict the development possible by the other team members. &
		To reduce the learning curve Dexter Lowe was chosen to lead Wowza development due to his familiarity with Java. Peter Wood has agreed to act as a mentor for Wowza development providing his phone number, email address and making himself available for meetings to discuss and give advice on Wowza development. \\ 
     \midrule
        Server loss &
        Moderate &
		Integral services will be hosted on two servers: The Windows Server hosting Wowza and the FTP service and the LAMP server for web development. If either of these become unavailable no further testing will be possible.  &
		All configuration information including Wowza configuration files and scheduled tasks information such as crontab will be stored on a source control system (Mercurial). This means that any server, including the personal machines of any team member may be quickly configured to minimise the impact on development. Additionally all configuration will be noted down such that if a server is corrupted any team member may quickly return it to defaults. \\ 
     \midrule
        Development machine loss  &
        Moderate &
		Any code, notes and documentation stored on the machine may be lost impacting the production schedule. &
		All code and documentation will be stored on a source control system (Mercurial) such that the developer need only move to a new machine to continue development. Additionally all members are in possession of at least one personal computer suitable for development and those provided by the university. This reduces the likelihood of a developer having no access to a development machine. \\ 
     \midrule
        Resources unavailable  &
        High &
		If a required resource is not available it may impede or halt development. &
		Alternatives to all resources such as redundant backups of data and extra machines will be attained where possible. The team will also search for ways to reduce reliance on resources. Additionally where unavailable resources are required for completion of an extension that extension will be regarded as out of scope. \\ 
     \midrule
        Resources unavailable (iPad)  &
        High &
		the project specification requires that the produced system work on the iPad but none of the team members own an iPad hence both development and testing may be difficult. &
		The University of Southampton has an iPad available from its stores for a maximum of a week. Our team will acquire special dispensation to keep the iPad for the duration of our project to avoid others taking the iPad during development. In addition extra iPads will be sought for each team member through the University to aid in testing and development. To prevent the unavailability of iPads impeding progression the streaming system will be developed for both the iPad using an HTML5 player and other devices using a Flash player to ensure that the team may continue development without an iPad in their possession by allowing them to use their computer browser for testing purposes. Due to the differences inherent in iPad development if each team member does not have an iPad then a member with an iPad will review their updates and ensure that they work. Additionally as much time collaboratively coding as is practical will be encouraged such that all members may have access to a single iPad. \\ 
     \midrule
        Requirement Misunderstanding &
        Moderate &
		If a member misunderstands their task they may waste time incorrectly implementing a feature. &
		All members have attained contact information for all other members. At least one two hour meeting per week will be attended by all members where tasks with be thoroughly talked through. In these meetings the elected group minute taker, Thomas Smith will take the minutes of each meeting and post them on a Google Document where members must review them to ensure understanding. Any information drawn or written in the meetings will be photographed and placed on the group to give the team members as much information as possible. Additionally all tasks are overseen by at least one other person reducing the likelihood of misunderstandings. Group coding sessions will be encouraged where practical placing as many members nearby as possible and finally regular commits will be required from which others can notice if a developer has misunderstood. \\ 
     \midrule
        Insufficient Recruitment for studies &
        Moderate &
		If volunteers cannot be obtained to test our hypotheses it may be difficult to prove our hypotheses.  &
		Studies will take place over several days, where possible not requiring volunteer presence. If recruitment is low the team will provide incentive to encourage participation funded by their personal finances. \\ 
     \midrule
        Unethical Studies &
        Low &
		If the team performs a skewed or unethical study, the results may be invalid and it may reflect poorly on the University of Southampton. &
		Ethics approval will be obtained through ERGO for all proposed studies. \\ 
     \midrule
        Network Outage &
        Moderate &
		As the project will be internet based and several sources are physically distant a network outage could prevent further development and testing as well as group communication. &
		No development will be performed on live servers hence in the case of network outage no data will be lost. Also where possible resources will be locally, redundantly replicated on developer machines to allow continued development elsewhere. \\
 \bottomrule
 \end{longtable}
 \end{landscape}
\end{center}
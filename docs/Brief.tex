\section{Brief}
\label{sec:appendix_brief}

The deliverable outcome of this project will be a prototype Channel 4 TV streaming web-application for the iPad, which will provide the viewer with a personal entertainment channel tailored to them specifically. The custom channel will consist of live streams from Channel 4, programs that are available via TV catchup and advertisements targeted towards the viewer. Programmes and streams will be available from Channel 4 and its sister channels: Film4, E4, More4 and 4Music.

While functionality exists in current TV streaming and catchup websites to recommend similar programmes, they are not integral to the service and simply offer a browsable list of shows. For example 4oD\footnote{\url{http://www.channel4.com/programmes/4od}} and iPlayer\footnote{\url{http://www.bbc.co.uk/iplayer}} offer a list of similar programs separate to the video player. User-specific data, such as demographics and viewing history, are not taken into account by these services.

In order to determine what programmes and adverts would be appropriate for the viewer, a diverse set of data sources will be utilised. By monitoring what people choose to view, we can target the viewer with programmes and adverts that complement their interests, or are trending at that time. We will capture personal data via social networks such as location information, age, gender and friends to further improve the targeting of programmes and adverts.

Viewers will be able to express a positive or negative attitude towards the currently displayed programme or advert - opting to skip it if it does not interest them. Upon expressing a negative attitude, the user will be asked to indicate why, so that their future playlist of programmes and adverts can be tuned. An EPG (Electronic Programme Guide) will be included in the application, allowing the viewer to browse live TV listings and non-live programmes on catchup, and add them to their own personal channel. A simple keyword search will also be offered such that a user can determine if and when a particular programme is available. 

Further work could include a friend system allowing a registered user to view other users' personal streams, with advertisement breaks remaining focused on the currently logged in user. Data regarding the viewing history of friends can also be used to tune which programmes are included in the current personalised playlist. Friends could possibly be inferred from linked social media accounts.

As an example use-case; consider a new user, John, a 21 year old male. He signs in to the service using his Facebook account. The listings for his personal channel can be seen, populated by programmes complementing his interests, including programmes which are popular within his demographics. John can bring up an EPG, which he can use to browse and select programmes to watch. As it comes closer to Valentine's day, John is shown more and more romantically themed adverts - but as he has just come out of a long-term relationship, he would rather these were not shown to him. By rating these adverts poorly, they are no longer shown, and are instead replaced by adverts for cars, which John prefers. Over time, as John chooses, skips and rates programmes, the programmes offered will more accurately reflect John's interests. 

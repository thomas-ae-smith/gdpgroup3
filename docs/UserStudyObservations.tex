\section{User study observations}
\begin{enumerate}
One non-interactive advert (Guinness) was provided in the pool of adverts used in the interactive session to simulate adverts where the advertiser did not wish to add any interaction. During this advert it was common for users to tap erratically on the screen scrolling around looking for interaction, some users even became irritated. One participant who received this advert first thought the system had broken and they were receiving another non-interactive session. This shows that if this system were to be used in production to avoid user frustration and confusion some form of interaction, even if only a like button would be needed on most adverts.

A significant portion of the users continued to tap on the screen intermittently after completing the provided interactive actions, when questioned about this they revealed that they thought other Easter eggs may be hidden in the advert. This inquisitive nature could be used by advertisers to encourage further attention by providing small hidden aspects to adverts for attentive viewers. For example a find 5 apples in the picture to get a coupon for 10\% off. This would encourage attention to the advert, especially if the task could not be completed in a single viewing in which case users may anticipate the advert.

Many of the participants suggested that a mini-game could be included, providing users with interaction for more of the advert and offering them the opportunity to compete with other viewers for the top spot, with the winner receiving a prize. This was a notable result simply for the number of people that suggested it, implying that it may be a popular idea, at least among the student demographic.

A large number of participants said that of all the adverts they remembered the Domino's one the most. When asked why most of them responded that it was due to the length of the advert. Long adverts can irritate users but many said that a longer interaction would have made it more palatable. This could allow advertisers to gain the advantages of long adverts without irritating the users by providing an extended interaction to keep enjoyment levels while maximising exposure.
\end{enumerate}
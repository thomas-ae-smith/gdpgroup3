\section{Project Proposal}
\label{sec:inqb8r_proposal}

Below is the project proposal by Peter Wood and Nick Culley of Inqb8r.

\begin{quotation}
\noindent The proposed project will be to create a modern TV viewing service for devices such as the iPad. 

``In band'' and ``out of band'' timed metadata are about to become the cornerstone of interactive TV offerings. Timed metadata allows for a host of complementary services to be added to the viewing device - interactive overlays, links to supplementary information, two-way live chatterboxing feeds (e.g., Twitter), popularity, etc. Timed metadata can also facilitate programme and advert replacement. ``Backwards EPG'' TV catchup viewing is also part of the proposition. The viewer could then create custom playlists in a ``Spotify for TV'' type fashion.  

This project would be facilitated and supported by Inqb8r, using its current `Project4' backend. Project4.tv is a Channel~4 streaming service provided to students at UK academic institutions over the JANET network. Project4 is a joint venture between Channel~4 and Inqb8r.

The current delivery methods, whilst fit for their current purpose, are not in line with modern and future web technologies. Inqb8r would like a group of the best and brightest to help shape the future of interactive internet/mobile television.

The HLS streams and data sources from Channel~4 are all in place and the student team will be granted access. Backend technologies include a mixture of industry standard encoders, media servers, and some bespoke hardware. Students will use a broad range of programming languages: for the backend to this service languages such as PHP, Java, MySQL, ActionScript, XML, Python. For the frontend: JavaScript, HTML5, CSS3, JSON will all need to be used.

BBC R\&D have done a fair bit of research into ``out of band'' timed metadata, with their LIMO project, and it is hope there will be some involvement from them as well but this is still to be confirmed. Other commercial projects of note include the second screen electronic programme guide `Zeebox' as well as the New~York based `Aereo' iPad TV service.

In essence the project is to create an iPad browser-based TV platform (frontend) for Channel~4's offerings (Channel~4, E4, More4, Film4, 4seven). We hope this proposal represents an exciting opportunity for ECS students to enhance their already diverse skill set, whilst working with national broadcasters and creating a tangible functional end product.

\vspace{5 mm}
\begin{centering}
	\textit{Inqb8r.tv}\\
	\textit{22 Soho Square, London, W1D 4NS, London}\\
	\textit{pete.wood@inqb8r.tv, nick.culley@inqb8r.tv}\\
\end{centering}
\end{quotation}
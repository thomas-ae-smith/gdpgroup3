\section{Evaluation}
	\subsection{Summary}
		In this project, we have built and tested a platform for interactive adverts which may be targeted towards users with detailed granularity. As part of this, we have developed a tool that allows the production of interactive adverts, and the creation of campaigns - targeting criteria for adverts - as well as viewing the statistics pertaining to these.

		We have demonstrated the use of this platform by developing a TV streaming service, your4.tv, which shows users programmes that are recommended to them based on their age, gender and occupation. Within this service, we use programme information and user information as input to our advert targeting platform, resulting in adverts that are more likely [back this up] to be of interest to the user. We target this service towards tablet computers, which allows interaction with adverts in a diverse and intuitive way through touch input.

		In the design stage of this project, three major subsystems were deduced. First, the advertising platform itself would operate on the web architecture. Data about adverts and campaigns would be transferred via a REST endpoint which could be used by both the client for an advert publisher and a viewer. Furthermore, this would allow the publisher to create their interactive overlays using HTML, CSS and JavaScript, allowing easy integration with other services such as social networks.

		Secondly, the adverts would need to be streamed. HTTP-Live Streaming (HLS) and H.264 are supported by HTML video on iPad. Wowza was an appropriate tool for streaming adverts.

		For your4.tv, a recommender system was built for 
	\subsection{Limitations}
	\subsection{Further Work}
		\subsubsection{Programme Recommendation}
		\label{sec:further_work_recommender}

		Currently, programmes are assigned binary vectors within $\mathcal{P}$; programmes either do or do not belong in each of the 18 genres. As a result, programme vectors may lie only in the corners of the hypercube geometrically representing $\mathcal{P}$. A logical improvement to the recommender system would be to allow fuzzy genre memberships, allowing programme vectors to exist anywhere within $\mathcal{P}$ and hence allowing fine-grained differences between similar programmes to be properly represented in the system.
		
		To initialize a programme with a fuzzy programme vector, \texttt{get\_programme\_vector} will be required to make use of more information than the current list of programme genres. Possible avenues to explore could include modifying $\mathcal{P}$ such that points are represented by genres pulled from multiple sources and reduced to a lower dimensionality feature space, where the feature space dimensionality would be set to minimise the number of dimensions while maximising the retained information. If additional external information processing is undesirable, user rating data could be used to modify programme vectors which are initialised as binary, although this is only useful in the case of recommending non-live programmes due to the cold start problem \citep{cold-start-problem}.

		If non-binary programme vectors are introduced to $\mathcal{P}$, a change is required in how a user vector is modified upon a negative programme rating. Under the current architecture, a user is pushed away from the vector of a negatively rated programme; if programme vectors exist away from the vertices of $\mathcal{P}$, a user who's vector somehow ends up at a vertex describing programmes they dislike will be unable to move away from the vertex by giving negative ratings, leaving them stuck. This is a difficult problem to solve and is outside the scope of this project; repeated bad ratings must not converge to a single point but explore the programme space, but must not pull a user vector away from a known `good' area. While a jump with random direction may work, storing the users previous rating will allow for exploitation of rating gradients, enabling use of more complex gradient-climbing techniques.

		The addition of implicit user ratings will remove the burden from the user of providing explicit ratings \citep{implicit_indicators}, especially if the rating interface is visually de-emphasised or removed entirely. Few interactions are offered by your4.tv from which to infer ratings from, but programme skipping is provided which carries implicit preference information \cite{exploiting_implicit_feedback}. \citep{recommender-systems-handbook}[p.~304] describe an implicit programme rating $\hat{r}$ calculated from the time a user has watched the programme $p$ and the total programme length $L$:
		$$
			\hat{r} = 3 + 2 \frac{t - 5}{L - 5},\quad 5 \leq t \leq L
		$$
The rating given is between 3 and 5, on a rating scale of 1-5, where play times of under 5 minutes are discarded, and a rating approaches 5 as more of the programme is watched. The implicit ratings gleaned from programme skipping may be made more reliable by asking the user why the programme was skipped; this would allow the rating calculating to consider whether the user does not like the programme, or if the skip was due to another reason (they may have already seen the programme elsewhere). Although this reintroduces the problem of breaking the users pattern of activity, it may be the case that the mental cost to answer why they pressed skip is less than that of giving a rating explicitly.

		%% I think the architecture has changed so this won't work anymore.
		% The current architecture of delivering recommendations to users can be modified to allow for much greater scalability. Currently, each user requests a recommendation and the server responds. If the userbase becomes large enough for this to become infeasable, one solution is to maintain a table of user vector cluster centroids which are updated periodically by a cronjob. Periodically, this table of centroids would be multicast out to all users, along with recommendations for each centroid. 
	\subsubsection{Advert Recommendation}

	If recommendation techniques are implemented to improve advert relevance, the data already being collected on user preferences will be of great value in predicting adverts the user will enjoy/engage with. Techniques have been developed \cite{contextual_advertising} which utilize this preference information, along with demographic \cite{contextual_advertising} information, which is also collected by your4.tv. A third data source utilised by the recommender system described by \cite{contextual_advertising} is user viewing histories, which your4.tv does not currently collect, though has potential to improve not only advert recommendations, but also programme recommendations, as mentioned in Section~\ref{sec:further_work_recommender}.

	\subsubsection{Waiting time/Recommendation quality trade-off}
	In the current system, programme playlists are constructed in a greedy fashion, using the following algorithm:
	\begin{algorithmic}[H]
	\State start\_timestamp $\gets$ now();
	\State playlist $\gets$ $[]$;
	\While{$($totalTime(playlist) $<$ 7200$)$ \textbf{or} $($len(playlist) $<$ 4$)$}
		\State P $\gets$ $[$all live programmes starting between startTime and startTime+300$]$;
		\If{P not empty}
			\State next\_programme $\gets$ best programme in P;
		\Else
			\State next\_programme $\gets$ best non-live programme;
		\EndIf
		\State playlist.append(next\_programme);
		\State start\_timestamp $\gets$ start\_timestamp + length(next\_programme);
	\EndWhile
	\end{algorithmic}
	This is far from optimum, as programmes with extremely high predicted ratings may start at slightly over 5 minutes after the previous programme ends, leading the programme not to be recommended. By viewing playlist-building as a problem of maximizing predicted programme ratings, minimizing gaps between programmes and prioritizing live TV, this can be seen as a nontrivial constrained optimisation problem, where the trade-off values would need to be determined empirically.

	\subsubsection{Streaming enhancements}

	Since adverts will be replaced by ones recommended to viewers, it must be known when adverts are due to start and finish. This process is achieved by Project4 using a GPI\citep{SCTE104} triggered hardware solution utilising an SCTE 104 Server, a Tanberg Encoder and a PV1000 Ad replacement device as described in \ref{subsubsec:Project4Tech}. 

	Ad break information is sent to a Miranda Xplayer Server from Channel 4 via an automation system which forwards this information as XML to a SCTE 104 device. This information is received via HTTP when the SCTE server is running in HTTP Server mode. Ad information is received on this interface 1 minute and 8 seconds before the Ad break is due to start. On receiving this information it checks with a PV1000 substitution device for a matching Ad break, see \citep{PV1000Schedule}. If it finds one it starts listening for a GPI pulse, sent from the automation system which when triggered instructs the Tanberg Encoder which encodes the video stream to insert markers around the ad breaks. This pulse is received 8 seconds before airing. The PV1000 devices then detect these markers and replace the ad break with one they have created\ref{Project4EventFlow}.

	The PV1000 receives its schedule from the TMS server. The TMS server receives EPG information from the princess server which pulls it from an EPG ftp service. A user uploaded Channel 4 schedule is also provided to the TMS server 2 weeks in advance. The calculated ad breaks are then provided to the PV1000.

	Without a GPI trigger millisecond precision is not possible for advert timings but assuming the network delay remains approximately constant second accuracy should be possible in most situations. Unfortunately live shows can result in these times being slightly out so in some cases the information may be incorrect. We have been given access by Inqb8r to the SCTE 104 control scripts which are coded in Python. Using these we should be able to minimise this risk by pinging the Wowza server and the LAMP server when the GPI pulse is received which is precisely 8 seconds before ad ad break is due to start. Additionally this information could be used to start and stop the recordings on the Wowza server resulting in recordings excluding the adverts allowing potential insertion of adverts at the user's discretion, allowing them to choose when to watch the adverts. However due to the lack of an available development device and the difficulty of modifying the scripts to perform our tasks without interfering with the existing Project4 tasks it has been decided that using this information is outside the scope of this project.

	\subsection{Conclusion}
In summary our project improves the user experience by making live media personal, targeting both shows and adverts to each user as well as integrating seamlessly with the social experience. Additionally it reduces frustration at missed shows as these are transparently skipped in favour of something the user can watch from the start. In addition it provides a zero-interaction discovery system to remove the time consuming hunt for something to watch and perhaps aid users in discovering new and exciting content. Moreover it improves the experience of consuming adverts by providing users with interactivity, personally targeted adverts and the ability to control which adverts they are shown. 

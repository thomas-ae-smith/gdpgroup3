\section{Evaluation}
	\subsection{Summary}
		In this project, we have built and tested a platform for interactive adverts which may be targeted towards users with detailed granularity. As part of this, we have developed a tool that allows the production of interactive adverts, and the creation of campaigns - targeting criteria for adverts - as well as viewing the statistics pertaining to these.

		We have demonstrated the use of this platform by developing a TV streaming service, your4.tv, which shows users programmes that are recommended to them based on their age, gender and occupation. Within this service, we use programme information and user information as input to our advert targeting platform, resulting in adverts that are more likely [back this up] to be of interest to the user. We target this service towards tablet computers, which allows interaction with adverts in a diverse and intuitive way through touch input.

		In the design stage of this project, three major subsystems were deduced. First, the advertising platform itself would operate on the web architecture. Data about adverts and campaigns would be transferred via a REST endpoint which could be used by both the client for an advert publisher and a viewer. Furthermore, this would allow the publisher to create their interactive overlays using HTML, CSS and JavaScript, allowing easy integration with other services such as social networks.

		Secondly, the adverts would need to be streamed. HTTP-Live Streaming (HLS) and H.264 are supported by HTML video on iPad. Wowza was an appropriate tool for streaming adverts.

		For your4.tv, a recommender system was built for 
	\subsection{Limitations}
	\subsection{Further Work}
		\subsubsection{Programme Recommendation}
		\label{sec:further_work_recommender}

		Currently, programmes are assigned binary vectors within $\mathcal{P}$; programmes either do or do not belong in each of the 18 genres. As a result, programme vectors may lie only in the corners of the hypercube geometrically representing $\mathcal{P}$. A logical improvement to the recommender system would be to allow fuzzy genre memberships, allowing programme vectors to exist anywhere within $\mathcal{P}$ and hence allowing fine-grained differences between similar programmes to be properly represented in the system.
		
		To initialize a programme with a fuzzy programme vector, \texttt{get\_programme\_vector} will be required to make use of more information than the current list of programme genres. Possible avenues to explore could include modifying $\mathcal{P}$ such that points are represented by genres pulled from multiple sources and reduced to a lower dimensionality feature space through Principal Component Analysis, where the dimensionality of the feature space would be set to minimise the number of dimensions while maximising the retained information. If additional external information processing is undesirable, user rating data could be used to modify programme vectors which are initialised as binary, although this is only useful in the case of recommending non-live programmes due to the cold start problem\citep{cold-start-problem}.

		If non-binary programme vectors are introduced to $\mathcal{P}$, a change is required in how a user vector is modified upon a negative programme rating. Under the current architecture, a user is pushed away from the vector of a negatively rated programme; if programme vectors exist away from the vertices of $\mathcal{P}$, a user who's vector somehow ends up at a vertex describing programmes they dislike will be unable to move away from the vertex by giving negative ratings, leaving them stuck. This is a difficult problem to solve and is outside the scope of this project; repeated bad ratings must not converge to a single point but explore the programme space, but must not pull a user vector away from a known `good' area. While a jump with random direction may work, storing the users previous rating will allow for exploitation of rating gradients, enabling use of more complex gradient-climbing techniques.

		%% I think the architecture has changed so this won't work anymore.
		% The current architecture of delivering recommendations to users can be modified to allow for much greater scalability. Currently, each user requests a recommendation and the server responds. If the userbase becomes large enough for this to become infeasable, one solution is to maintain a table of user vector cluster centroids which are updated periodically by a cronjob. Periodically, this table of centroids would be multicast out to all users, along with recommendations for each centroid. 
	\subsubsection{Advert Recommendation}

	If recommendation techniques are implemented to improve advert relevance, the data already being collected on user preferences will be of great value in predicting adverts the user will enjoy/engage with. Techniques have been developed \cite{contextual_advertising} which utilize this preference information, along with demographic \cite{contextual_advertising} information, which is also collected by your4.tv. A third data source utilised by the recommender system described by \cite{contextual_advertising} is user viewing histories, which your4.tv does not currently collect, though has potential to improve not only advert recommendations, but also programme recommendations, as mentioned in Section~\ref{sec:further_work_recommender}.

	\subsection{Conclusion}
In summary our project improves the user experience by making live media personal, targeting both shows and adverts to each user as well as integrating seamlessly with the social experience. Additionally it reduces frustration at missed shows as these are transparently skipped in favour of something the user can watch from the start. In addition it provides a zero-interaction discovery system to remove the time consuming hunt for something to watch and perhaps aid users in discovering new and exciting content. Moreover it improves the experience of consuming adverts by providing users with interactivity, personally targeted adverts and the ability to control which adverts they are shown. 
\section{Prestudy}
\label{sec:prestudy}
In order to inform the development of the project, and provide a reasonable basis for some of the design decisions made about the system, an initial prestudy was run with questions covering a wide range of relevant areas. The primary intention of the survey was to find out how people currently consume %TODO: specify what kind of media
media, and their responses to the adverts that support the services they use. The study also included questions about individual levels of acceptance of the use of personal information for advertisement targeting, and an opportunity to share further personal opinions and comments in an optional free text question. %TODO: summary sentence for this paragraph

The data collected during the survey 

%Performing this study to establish the best way to retain user attention to adverts, thereby making any product employing these techniques a more desirable option for advertisers resulting in greater profits for advertisers and greater user satisfaction.

%The aim of this research is to establish how students consume streaming media and the current view of advertising supplied with the media from the perspective of university students as well as determining typical student actions when advertising is served.

%We hope to use this information to inform our design of a new streaming application to maximise the number of adverts that a user will give their attention to thereby increasing advertiser and user satisfaction. We also believe that even if a user does not pay full attention to advert that there may still be some avenue for user contact. Determining what a user does when an advert appears will allow us to tailor our adverts to take advantage of other avenues of advert such as audio only and maximise our user contact platform.


Purpose of the study:\\
Collect data on how people consume media and adverts\\
Inform the development of features for the project\\
Elicit basic information about viewer habits

\subsection{Study methodology}

Methodology:\\
Performed early\\
Short and easy and anonymous\\
Webpage\\
Selected students as a sample - representative (find reference)\\
Protected results, ethics number 4315\\



The study was presented to a number of students during the first few weeks of the project, as the 

mainly students, the same as the intended audience


The target demographic for the project's final product is university students, and so it makes sense to approach a random sample of students directly. The survey design was intentionally short and minimal, to maximise the likelihood of agreement, we will also make it clear to any potential participant that no personally identifiable information will be collected and that they will not be contacted about their responses. %Talk about using an URL

The proposed sample are a random sample of the students attending the University of Southampton and friends.


To perform this study an anonymous questionnaire has been chosen as the best approach. This is because the information we gather will be used to tailor our service but will not require any further interaction on the part of the set of participants as they will not be provided with our product, only asked their opinion of existing and potential systems.



Talk about:\\
 - Purpose\\
 - Content\\
 - Sample\\
 - Ethics 4315\\
 - Methodology (gdocs)\\
 - Analysis\\
  - Specific interest\\
  - Resulting decisions\\


Format and frequency of media consumption\\
Response towards accompanying adverts\\
Acceptance of ad targeting with personal information\\
Free text question: interesting personal viewpoints\\





The prestudy page as presented to participants is available at \href{http://your4.tv/\#prestudy}{\texttt{your4.tv/\#prestudy}} and a listing of the questions can be found in Appendix~\ref{sec:appendix_prestudy}. %TODO: make this happen


\subsection{Survey outcome}

Results:\\
67 student participants\\
Discovered specific features we should focus on:\\
Relevant adverts\\
Ability to skip adverts\\
... suggests that the product we propose will improve the ...

Almost nobody actually pays attention to all of the adverts that they are shown. In particular, amongst viewers watching other media such as Youtube, more than half of them never watch any of the advertisements at all - from the free text responses, it seem this is largely people using ad blocking programs. On the other hand, a far larger proportion of viewers Always or Sometimes watch adverts on live TV, which is certainly good news for us.

Having seen that almost everyone ignores adverts sometimes, we also asked what people did instead of paying attention to them. Again, we can see that more than three quarters of the respondents skip adverts at least occasionally, where possible. We found that a few of the respondents mentioned `zoning out' and losing interest during ad breaks - we can also see that people use the advertisement time to do other tasks - generally on the same device, but less frequently on another one, or something completely different like making tea, etc. We use this data to help inform the development of functionality for the system.

Finally, we looked at which specific aspects affected why people were likely to watch an advert (or not), with some interesting results. The two factors with the largest influence were the relevance of the advertisement and whether or not there was opportunity to skip it - both of which are addressed in the service we are developing. We also found, through analysing the free text responses, that people have a generally poor opinion of adverts that force a response from the user, whether they be obnoxiously loud, or simply require the user to interact with them before delivering content. Again, we can refer to these findings to help us to design a product that people will actually use. 

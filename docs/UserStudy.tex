% TODO: Get a citation for better attentivity => better information recall.
\section{User Study}
	\subsection{Overview}
	A study was performed in which the change in user attentivity was measured upon being presented with interactive and non-interactive advertisements. To estimate user attentivity, two metrics known to correlate well with attentivity were used: information recall\cite{} and time perception\cite{yahoo-intrusive-advertising}.

	\subsection{Methodology}
	Using a set of 119 student targetted adverts taken from Project4, \large{\textbf{N}} were given interactive html/css overlays containing interactive content likely to be typical for an interactive TV advert. Each participant in the study was shown two rounds of adverts---one interactive and one non-interactive---with half the participants being shown the interactive adverts first, and the other half shown non-interactive adverts first. Participants were shown a total of \large{\textbf{10}} minutes of adverts, but were split into three groups each with different ratios of interactive/non-interactive advert showing times: group 1 had a 5/5 split, group 2 a 4/6 split and group 3 a 6/4 split. 

	\begin{table}[hb]
		\centering
		\begin{tabular}{ c l }
			\toprule
			\bf Product & \bf Interactive content \\
			\midrule
			Pot Noodle & Poll: The user is encouraged to vote for their favorite flavour of pot noodle. \\
			Smirnoff Vodka & Information: A user may bring up an interactive element where they may view a number of cocktails containing Smirnoff Vodka, and are encouraged to visit the product website. \\
			Movie & Information: The user may enter their postcode prompting a map to be expanded showing nearby cinema showings. \\
			Scotland Holidays & Information: The user may enter their email address to request more holiday-related information. \\
			\bottomrule
		\end{tabular}
		\caption{Interactive content added to advertisements}
		\label{tab:interactive_content}
	\end{table}

	

	\subsection{Results}

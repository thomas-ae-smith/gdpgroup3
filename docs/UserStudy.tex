% TODO: Get a citation for better attentivity => better information recall.
\section{User Study}
	\subsection{Overview}
	A study was performed in which the change in user engagement was measured upon being presented with interactive and non-interactive advertisements. To estimate user engagement, two known correlating measurements were taken: information recall\cite{} and time perception\cite{yahoo-intrusive-advertising}.

	\subsection{Methodology}
	Using a set of 119 student targetted adverts taken from Project4, 8 were given interactive html/css overlays containing interactive content likely to be typical for an interactive TV advert. Each participant in the study was shown two rounds of adverts---one interactive and one non-interactive---with half the participants being shown the interactive adverts first, and the other half shown non-interactive adverts first. The length of the rounds varied between 4, 5, and 6 minutes, and were evenly split between longer interactive rounds, longer non-interactive rounds and equal length rounds; the purpose of which was to measure differences in the participants perceived time and actual round time. 

	The study was scripted, which has been included in Appendix~\ref{sec:appendix_interview_script}. While participants were neither encouraged nor discouraged from interacting with the interactive set of adverts, they were informed that interactions were possible, and how to interact, removing the initial learning period. As part of the study, an video introduction was given, which is included in Appendix~\ref{sec:appendix_video_script}.

	\begin{table}[hb]
		\centering
		\begin{tabularx}{\linewidth}{ c X }
			\toprule
			\bf Product & \bf Interactive content \\
			\midrule
			Pot Noodle & \textbf{Poll}: The user is encouraged to vote for their favorite flavour of pot noodle. \\
			Smirnoff Vodka & \textbf{Social}: The user may `like' the product, interacting with the social networking website Facebook. \\ % TODO: do.
			Movie & \textbf{Information}: The user may enter their postcode prompting a map to be expanded showing nearby cinema showings. \\
			Scotland Holidays & \textbf{Information}: The user may enter their email address to request more holiday-related information. \\
			Fosters Lager & \textbf{Social}: The user may `like' the product, interacting with the social networking website Facebook. \\
			Unified Insurance Cover & \textbf{Information}: The user may select possessions they own, and are given an estimated insurance quote. \\
			Dominoes Pizza & \textbf{Purchase}: The user may interact witht he advert to order a pizza. \\
			\url{www.thetrainline.com} & \textbf{Poll}: The user may vote whether or not they think train tickets are too expensive, and are shown the poll results upon voting. \\
			\bottomrule
		\end{tabularx}
		\caption{Interactive content added to advertisements}
		\label{tab:interactive_content}
	\end{table}

	\subsection{Results}

	Analysing the answer to question 2b (ii), ``\textit{Please estimate how much time more or less you were watching in the second session?}'', shows how users perceived time during the live and non-live rounds. 


	% TODO: Make some bell curves and stuff.
	%\begin{figure}[hb]
		%\centering
		%\begin{subfigure}[b]{0.5\textwidth}
			%\centering
			%\includegraphics[width=\textwidth]{}
			%\caption{Time overestimate on non-interactive adverts}
		%\end{subfigure}
		%\begin{subfigure}[b]{0.5\textwidth}
			%\centering
			%\includegraphics[width=\textwidth]{}
			%\caption{Time overestimate on interactive adverts}
		%\end{subfigure}
		%\caption{}
	%\end{figure}

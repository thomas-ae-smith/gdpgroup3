\section{Motivation}
With the increasing pervasiveness of mobile devices noted by \citet{socialTV} there is surprisingly limited options for tablet users to access live streaming television. Moreover the nature of the interactive platform provided by these devices, opens up diverse and interesting interactivity possibilities for exploiting physical interaction to hold viewers attention, even during adverts they may otherwise have ignored. Furthermore the disparity in the number of viewers per viewing device between traditional televisions and mobile devices could be constructively exploited to provide granularly targeted adverts. In this section we will first introduce the customer before examining the extra functionality of tablet devices. We will then consider the issues raised by the continuing evolution of television and some of the systems that have arisen to help deal with these issues. Finally we will consider the usefulness of interactivity and granular advert targeting in improving viewer attentiveness to adverts and the relative impacts on the advertisers and users of a system utilising these features.


\subsection{Inqb8r}

	Inqb8r\footnote{\footurl{http://inqb8r.tv}}, the primary client for this research project, is a UK based business which provides a platform for content owners and advertisers alike to expose their own products and services to a university student focused audience. Inqb8r also provides content creation services to their customers in order to aid them in maximising their contact with students.
	
	One of Inqb8r's primary products is Project4\footnote{\footurl{http://www.project4.tv}}. Project4 is TV streaming service which offer student focused content and advertisements. This addressable advertising service offers an invaluable opportunity for advertisers to increase student audience reach.
	
	Due to the success of their platform Inqb8r are very interested in expanding the reach of Project4 and leverageing the additional features of tablet devices. At present most of their users access their service via a web browser. This is in part due to their use of a Flash based player which is not compatible with iPads and other apple devices.
	
	With the advert of HTML5 Inqb8r are very interested in the utilisation of new HTML5 technologies to open up their service to a wider audience. Furthermore as their software utilises an interactive overlay for basic control they are interested to see how interaction centric platforms such as iPads can be leveraged to provide greater viewer interest. They are particularly keen to see if integrating traditional second screen technologies directly into the viewing medium has an impact on usability and enjoyability. 

\subsection{Live Streaming}
The iPad and other tablet devices have for some time had apps like 4oD available which allows on demand streaming of recorded Television content but the main UK channels have limited provision for live streaming. Channel 4 provided a Paralympics app\footnote{The official Channel 4 Paralympics Real-time Commentary App\url{https://itunes.apple.com/gb/app/channel-4-paralympics/id554157549}} which promised live commentary. Unfortunately this only provided short videos (lacking sound) and text commentary with images much like a Twitter stream or blog.

The BBC iPlayer App provides a limited subset of live streams and Sky subscribers can use Sky Go to get a rather restrictive subset of the Sky channels. This limited uptake is surprising considering that \citep{socialTVPaper} argues that tablets are becoming a major portal for television viewing and several third parties are taking advantage of this such as \footnote{TVCatchup Live TV App: \url{http://www.tvcatchup.com/}}.

Traditional costings analyses such as the Spence-Owen analysis of pay TV assumes that the cost of adding another viewer is 0\citep{broadcastEconomics}. While this is clearly the case for traditional radio broadcast as there is no two way communication required, internet devices such as tablets do have a cost.

Most traditional streaming services provide a unicast stream to each viewer which means that the load placed on the servers increases linearly resulting in a linear increase in costs as more users connect. \cite{cachedStream} realised that this does not scale well and proposes a different architecture using self organising caches and stream segmentation. \citep{cachedStream} concludes that their proposed helper decreases load significantly so using a system such as that proposed with a protocol such as Apple's HLS\cite{HLS} which segments the stream as part of the protocol the cost of adding a user can be significantly reduced however it still does not meet the zero requirement noted in \citep{broadcastEconomics}.

Another internet broadcast possibility is using multicast technology\citep{multicast}. Multicasting reduces server load significantly as the source need only forward the stream to one endpoint, the multicast group. This means that adding new listeners does not negatively impact server performance hence giving the 0 additional cost required for the Spence-Owen analysis\citep{broadcastEconomics}.

While mobile device usage continues to rise uptake of this new medium is slow. As many of the main channels such as Channel 4 provide on demand services such as 4oD. They will be well aware of the costs of unicast services and are unlikely to want to expand into this new area unless it can be proven economically viable. Systems such as Project4 attempt to improve profitability by replacing the adverts with ones targeted towards their user demographic of students while keeping costs low using multicast technology. By extending this idea to providing individual granular targeting we can infer that advertisers would be able to reach a greater percentage of their target audience with less advert showings but this unfortunately would require reverting to unicast as each user's stream would be different. Clearly a hybrid approach using multicast where possible to minimise unicast traffic would maximize the utility of tablet devices to broadcasters. Additionally, using an easily cacheable segmenting protocol such as HLS\citep{HLS}, with a helper such as that proposed in \citep{cachedStream} would maximise individuality while minimising server load and hence costs.

\subsection{Television Format}
Traditionally users watch shows on a given channel in isolation and after each may switch to a different channel to watch another programme. This means adverts can be focussed towards the viewers likely to be watching a given show which is far easier than predicting the average viewership of an entire channel. However in-between two shows and when a user first turns on the television their attention is taken by the search for something to watch.

If a given user finishes watching a show on channel c and then wishes to watch a show on a different channel c' they must spend a length of time t navigating from c to c'. As the user's desired location is c' advertisers may be targeting the user who is not yet present on c' by predicting that those watching for the upcoming show would be interested in their advert. Therefore if an advertiser pays for an advert spot of length t, during which time the user is transitioning, then that user will not see the advert. This reduces the amount of viewers who will see that advert.

If the requirement to search was lifted then this would provide a greater frequency(marketing term referring to the number of times a given viewer will see an advert) for a given value of investment. Research has shown that an OTS(Opportunities to see) value of 3 gives maximum ROI(return on investment) so maximising the number of viewers per spot means that to achieve an OTS of 3 an advertiser may need to pay for less actual spots to achieve the same number of viewers with an OTS of 3 allowing them to invest this money in other areas. Additionally the user experience would be improved by removing the tedious search and minimising the amount of non-viewing time in a given usage period.

These days whether a user has Freeview, Sky or even just standard terrestrial television they have a multitude of channels to choose from this results in an unrealistic amount of information for users to consider\citep{informationOverload}. Even under the banner of what were once single channels like Channel 4 or Sky rest a considerable number of actual channels each of which has TV listings information. For example there are currently five major channels under the brand of Channel 4\footnote{Channel 4, E4, More4, 4Music and Film4}. This allows the broadcaster to provide more content, improving their reach at a given time as they are no longer restricted to a single market segment as each channel's content may be targeted towards a different group allowing a broadcaster to provide appealing content to a greater market segment. While modern systems allow EPG data to be displayed and searched on the TV the fact that TV is live means that by the time they have located a desirable show it may already have started. This is somewhat resolved by time-shifted channels which are simply rebroadcasts of other channels at a later time such as E4+1. Some services have also started offering 2 hour shifted channels but these still require a user to be present and ready at the exact time of broadcast. If users are more able to move to their next optimal channel while minimising the transit time t this will mean they see more relevant adverts in a given usage period as they would be on a channel showing appealing content for a greater percentage of their time, hence if advertisers target their adverts to the shows well, then improving this situation could significantly improving the OTS values of relevant adverts.

Some more advanced systems allow time-shifting such as TiVo allowing users to pause and rewind live TV. It achieves this by recording a rolling 30 minutes of the last 2 channels the user was watching\footnote{Description of TiVo's Live TV rewinding technology\url{http://www.mytivo.com.au/whatistivo/tivois/pauserewind/}}. So even these systems which are considered advanced have their limitations.

Having missed their show users then have two choices, hope that it will be rebroadcast on TV or use a TV catchup service such as 4oD. If the user chooses to use an on demand service then they likely have to sort through even more information as most catchup services provide at least the previous 7 days of content.

Having considered these limitations and our resources, which included all the Channel 4 stream sources, we approached the issue of missing the start, or locating an engaging programme in a different way. We decided to provide the user with only one channel which would be generated when the user accesses our system by using what is known about them to build a playlist from all the live channels our system knows of, if there is nothing the user would find appealing on, or they have missed the start it will fall back to adding pre-recorded shows to the playlist until something appropriate started. As shows that have started are transparently skipped the user may miss a desired programme but because their channel consists of a mixture of live and recorded TV, once the show has been recorded it may be added to their playlist at a time when there is nothing engaging on live TV, meaning that users are able to watch without fear of missing shows at a time that suits them without the hassle usually involved in the process of locating a show.

\subsection{Adverts}
Adverts generate revenue for the broadcaster and importantly are embedded in the stream at the source. This means that there is traditionally very little in the way of customisation. Advertisers pay for advert spots when they believe their target audience will be watching but it is unlikely that all of the viewers will be interested in the adverts. The user experience of any viewers who are not interested in the advert is negatively impacted in these situations. Some services such as YouTube have added the ability to skip their ads. This improves their overall user experience as if they find an advert annoying or offensive they have the ability to skip it. Furthermore other systems such as Facebook allow users to close adverts but request that the user says why they chose to close it. This provides useful feedback, helping to prevent inappropriate adverts and allowing advertisers to improve their future advertising campaigns.

Unfortunately as the adverts are inserted at the source it is difficult to allow users of the TV medium to skip adverts when they are shown live. Although as user's may now record television they may simply watch it later and fast forward through the adverts meaning that it is more important than ever to maximize the viewer's enjoyment of adverts to reduce the likelihood of them using these functions.

\subsubsection{Advert Relevance and Control}
Despite the difficulty of skipping adverts systems such as Project4 do replace these adverts with more targeted ones through the use of triggered hardware specifically the PV1000 triggered by a GPI pulse, but due to the difficulty of synchronisation\cite{softwareTimeSync} it is very hard to achieve this in software. This is an important distinction because the hardware solutions are impractical for individual targeting as every user would need a separately ad-replaced stream. As much as users may desire it there is no way to skip an ad-break altogether. This is because the rest of the programme has not yet been broadcast so there is no media to return to. Therefore improving the user experience in current systems is limited to improving the relevance of adverts.

We hypothesise that users would be more likely to enjoy an advert if it were relevant to them, with this assumption we can infer that personally recommended adverts targeted to each user would improve user interest, thus improving the user experience, whilst simultaneously providing advertisers with greater utility per impression, by maximising the likelihood that a viewer will be interested in their product. Additionally allowing the users to blacklist adverts should improve the user experience by allowing users to control which adverts they see and if they provide a reason for their disapproval, this is valuable feedback for advertisers to allow them to improve future adverts.

\subsubsection{Interaction and Engagement}
We also hypothesise that if a user is less engaged by an advert then they may ignore it or even leave the room giving advertisers little or no utility and only annoying the users. Most current adverts do not take advantage of the interactive platform provided by tablet devices. There are second screen applications such as Shazam\cite{shazam} which allow users to perform related tasks such as tagging the adverts which they see, which has been shown to increase brand recall but involves performing a supplementary task drawing attention from the advert. Providing the adverts interactively would reduce the requirements as only one device would be needed. This would also allow advertisers to provide useful or fun interactive content to reinforce the advert while still keeping partial attention on the original advert.

By keeping the user focussed on the advert even if they do not pay direct attention to the content interaction provides another chance for users to remember the advert and provides greater benefit in advert repetition as users may interact with the advert even if they have seen the content before if it is entertaining or useful. For example in an advert for a film location data could be used to show the closest showings. When the user first sees the advert they may not be looking for a film, but later at the prompting of the advert they then have an instant interaction which may allow them to discover a showing at a local cinema starting soon thereby eliminating the search on a cinema's website reducing the length of the funnel required for a conversion.

\subsection{Individual Targeting}
\citep{socialTVPaper} argues that television is defined by the content that is viewed, not the device it is viewed on and that tablets are fast becoming one of the main devices people use to consume television. iPads and indeed most portable devices such as smart phones, tablets and laptops are typically used by a single person. Because of this many personal systems are already tightly integrated such as social media, like Twitter or Facebook. Additionally while a standard television may be watched by a whole family or more, anything watched on a tablet will likely be viewed by a single person, the owner. This means that adverts can be targeted individually to members of a family rather than trying to cover the full spectrum.

Social interaction such as the action of liking something on Facebook can be easily integrated into adverts allowing users to save information on their personal profiles for later reference. This is useful in brand recall. For example if a user likes a brand that brand's Facebook posts will appear in their time-line reminding them of the brand and additionally providing a reference if they are having trouble recalling the product they saw advertised.

\citep{socialTVPaper} further argues that television on the web are converging more and more to the point where the social experience has direct impact on what is broadcast as users can respond to what they see in the social sphere for example by Tweeting at a show or broadcaster. By extending the conclusions of \citep{socialTVPaper} we can infer that encouraging social participation in advertising would further allow users to control what they see and provide rich feedback to the advertisers which is confirmed by \citep{socialTV}.

Systems such as ``Apple Airplay''\footnote{Technology to stream media to alternate devices: \url{http://www.apple.com/airplay/}} can allow users to use their personal device as a source, and transmit media to a compatible endpoint device. This allows users to display something on their personal device on a more public device such as a television which can cause issues with individually targeted services.
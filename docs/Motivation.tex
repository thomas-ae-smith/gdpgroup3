\section{Motivation}

\subsection{Live Streaming}
The iPad and other tablet devices have for some time had apps like 4oD available which allow on-demand streaming of recorded television content but there have been very few attempts by the major TV channels to bring live TV streaming to these devices. Channel 4 provided a Paralympics app\footnote{\url{https://itunes.apple.com/gb/app/channel-4-paralympics/id554157549}} which promised live commentary. Unfortunately this only provided short videos (lacking sound) and text commentary with images much like a Twitter stream or blog.
%TODO: needs concluding sentence

\subsection{Television Format}
Traditionally users watch shows individually a channel. This allows adverts to be focussed towards the viewers likely to be watching a given show. However in-between two shows and when a user first turns on the television their attention is taken by the search for something to watch. This reduces advert utility for the advertisers as some of the viewers who could be watching their adverts are engaged with their search. If the requirement to search was lifted then advertisers would have a greater number of engaged viewers improving the utility of their advert. Additionally the user experience would be improved by removing the tedious search and minimising the set up time for a viewing session.

These days whether a user has Freeview, Sky or even just standard terrestrial television they have a multitude of channels to choose from - this results in an unrealistic amount of information for users to consider\citep{informationOverload}. Even under the banner of what were once single channels like Channel 4 or Sky rest a considerable number of actual channels, each of which have TV listings information. For example there are currently five major channels under the brand of Channel 4\footnote{Channel 4, E4, More4, 4Music and Film4}. While modern systems allow Electronic Programme Guide (EPG) data to be displayed and searched on the TV the fact that TV is live means that by the time they have located a desirable show it may already have started. This is somewhat resolved by time-shifted channels which are simply rebroadcasts of other channels at a later time such as E4+1. Some services have also started offering 2 hour shifted channels but these still require a user to be present and ready at the exact time of broadcast.

Some more advanced systems allow time-shifting such as TiVo allowing users to pause and rewind live TV. It achieves this by recording a rolling 30 minutes of the last 2 channels the user was watching\footnote{\url{http://www.mytivo.com.au/whatistivo/tivois/pauserewind/}}. So even these systems which are considered advanced have their limitations. 

Having missed their show users then have two choices, hope that it will be rebroadcast on TV or use a TV catchup service such as 4od. If the user chooses to use an on demand service then they likely have to sort through even more information as most catchup services provide at least the previous 7 days of content.

Having considered these limitations and our resources, which included all the Channel 4 stream sources, we approached the issue of missing the start, or locating an engaging programme in a different way. We decided to provide the user with only one channel which would be generated when the user accesses our system by using what is known about them to build a playlist from all the live channels our system knows of, if there is nothing the user would find appealing on, or they have missed the start it will fall back to adding pre-recorded shows to the playlist until something appropriate started. As shows that have started are transparently skipped the user may miss a desired programme but because their channel consists of a mixture of live and recorded TV, once the show has been recorded it may be added to their playlist at a time when there is nothing engaging on live TV, meaning that users are able to watch without fear of missing shows at a time that suits them without the hassle usually involved in the process of locating a show.

\subsection{Adverts}
Another common issue with streaming media is the adverts. These raise money for the broadcaster and importantly are embedded in the stream at the source. This means that as much as users may request it there is no way to skip an ad-break. This is because the rest of the programme has not yet been broadcast so there is no media to return to. Having considered how users consume adverts we concluded that adverts are necessary but typically users do not currently enjoy consuming adverts.

\subsubsection{Advert Relevance and Control}
We decided that our system should maximise both user enjoyment and advertiser utility. By hypothesising that users would be more likely to enjoy an advert if it were relevant to them, we determined that our product should provide personally recommended adverts targeted to each user. This would make users seeing the adverts more interested improving the user experience whilst simultaneously providing advertisers with greater utility per impression by maximising the likelihood that a viewer will be interested in their product. Additionally we decided that a skip button allowing the user to blacklist adverts would improve the user experience by allowing users to control which adverts they see and if they provide a reason why they skipped, this is valuable feedback for advertisers to allow them to improve future adverts.

\subsubsection{Interaction and Engagement}
We hypothesised that if a user is less engaged by an advert then they may ignore it or even leave the room giving advertisers little or no utility and only annoying the users. By realising that current adverts do not take advantage of the interactive platform provided by tablet devices we predicted that by providing the adverts on an interactive plane we could maximise user attention by providing useful or fun interactive content to reinforce the advert. Furthermore by keeping the user focussed on the advert even if they do not pay direct attention to the content the interaction provides another chance for users to remember the advert and indeed provides greater benefit in advert repetition as users may interact with the advert even if they have seen the content before, if it is entertaining or useful. for example a showings finder embedded in an advert for a film. When the user first sees the advert they may not be looking for a film, but later at the prompting of the advert they then have an instant interaction which may allow them to discover a showing at a local cinema starting soon thereby eliminating the search on a cinema's website and maximising the chance that the viewer would see the advertised film over another which may have been shown on the cinema site.

\subsection{Personal Focus}
iPads and indeed most portable devices such as smart phones, tablets and laptops are typically used by a single person. Because of this many personal systems are already tightly integrated such as social media, like Twitter or Facebook. Additionally while a standard television may be watched by a whole family or more anything watched on a tablet will likely be viewed by a single person, the owner. This means that adverts can be targeted individually to members of a family rather than trying to cover the full spectrum, additionally social interaction such as the action of liking something on Facebook can be easily integrated into adverts allowing users to save information on their personal profiles for later referral.

\subsection{Summary}
In summary our project improves the user experience by making live media personal, targeting both shows and adverts to each user as well as integrating seamlessly with the social experience. Additionally it reduces frustration at missed shows as these are transparently skipped in favour of something the user can watch from the start. In addition it provides a zero-interaction discovery system to remove the time consuming hunt for something to watch and perhaps aid users in discovering new and exciting content. Moreover it improves the experience of consuming adverts by providing users with interactivity, personally targeted adverts and the ability to control which adverts they are shown. 